\documentclass{article}
\usepackage[margin=1in]{geometry}

\usepackage[%
  backend=biber,
  %style=numeric,
  sorting=none,
  minnames=15,
  maxnames=25,
  defernumbers=true,
]{biblatex}

\usepackage[ttscale=.875]{libertine} \usepackage[T1]{fontenc}

\usepackage{bold-extra}   % Provides bf+sc (only in textbf+textsc env.)
\usepackage{calc}         % Do math in tex (used for percent calculation)
\usepackage{comment}      % Provides \begin,\end{comment} for large blocks
\usepackage{longtable}    % Tables that spans pages
\usepackage{makecell}     % Multi-line table cells
\usepackage[protrusion=true,expansion=true,kerning,spacing]{microtype} % better type, spacing
\usepackage{tabularx}     % Complicated table creation
\usepackage{textcomp}     % Missing UTF characters
\usepackage[sc]{titlesec}
\usepackage{url}          % Pretty printing of hyperlinks
\usepackage[usenames,dvipsnames,svgnames]{xcolor} % Allow the use and definition of colors
\usepackage{xparse}       % For doing math (acceptance percentage in particular)
\usepackage{xspace}
\usepackage{xstring}

% The hyperref package must loaded last. Can conflict with some packages, see:
% README ( http://ctan.mackichan.com/macros/latex/contrib/hyperref/README.pdf )
\usepackage[colorlinks=true,citecolor=violet,urlcolor=Navy]{hyperref}     % Creates hyperlinks from ref/cite
\hypersetup{pdfstartview=FitH} % Sets default zoom to 100% width

% Break URLs properly (thanks to Alex Halderman)
\def\UrlBreaks{\do-\do\.\do\@\do\\\do\!\do\_\do\|\do\;\do\>\do\]\do\)\do\,\do\?\do\'\do+\do\=\do\#}
\def\UrlBigBreaks{\do\:\do\/}

% Some macros that a broadly useful:
\newcommand{\uW}{{\textmu}W\xspace}
\newcommand{\uA}{{\textmu}A\xspace}
\newcommand{\um}{{\textmu}m\xspace}
\newcommand{\us}{{\textmu}s\xspace}
\newcommand{\uF}{{\textmu}F\xspace}
\newcommand{\uJ}{{\textmu}J\xspace}
\newcommand{\iic}{I$^2$C\xspace}
\newcommand{\vdd}{V$_{\textnormal{DD}}$\xspace}

% Don't typset URLs in tt font
\urlstyle{sf}



% Assign labels based on source file
% https://mirrors.concertpass.com/tex-archive/macros/latex/contrib/biblatex/doc/biblatex.pdf
% §3.8.1 Bibliograpy Commands :: Resources

\addbibresource[label=CATbook]{books.bib}
\addbibresource[label=CATjournal]{journals.bib}
\addbibresource[label=CATconference]{conferences.bib}
\addbibresource[label=CATworkshop]{workshops.bib}
\addbibresource[label=CATposterdemo]{posterdemo.bib}



% Longtable adds more padding than tabular by default, change to match tabular's no-padding
\setlength{\LTpre}{0pt}
%\setlength{\LTpost}{0pt} -- Actually, bottom padding is good

% Configure makecell cells to be top-left aligned
\renewcommand{\cellalign}{tp{5.75in}}

% Custom bibtex keys, from
% http://tex.stackexchange.com/questions/111846/biblatex-2-custom-fields-only-one-is-working
\DeclareSourcemap{
  \maps[datatype=bibtex,overwrite=true]{
    \map{
      \step[fieldsource=acceptance-total]
      \step[fieldset=usera,origfieldval]
    }
    \map{
      \step[fieldsource=acceptance-accepted]
      \step[fieldset=userb,origfieldval]
    }
    \map{
      \step[fieldsource=acceptance-total]
      \step[fieldset=acceptance-accepted,append=true,fieldvalue=/]
      \step[fieldset=acceptance-accepted,append=true,origfieldval]
      \step[fieldsource=acceptance-accepted]
      \step[fieldset=userc,append=true,origfieldval]
    }
    \map{
      \step[fieldsource=extra]
      \step[fieldset=userd,origfieldval]
    }
    \map{
      \step[fieldsource=acceptance-percent]
      \step[fieldset=usere,origfieldval]
    }
  }
}

\ExplSyntaxOn
\NewDocumentCommand{\myMathFunction}{m}
{ \fp_eval:n {round((#1)*100)} }
\ExplSyntaxOff

\DeclareFieldFormat{userc}{\myMathFunction{#1}\%}

\AtEveryBibitem{%
  \csappto{blx@bbx@\thefield{entrytype}}{% put at end of entry
    \iffieldundef{usera}{%
%    \space \textbf{No annotation!}}{%
    }{%
      \\Acceptance:%
      \space\printfield{userb}~/~\printfield{usera}%
      \space(\printfield{userc}).
    }
    \iffieldundef{usere}{%
    }{%
      \\Acceptance:%
      \space\printfield{usere}\%.
    }
    \iffieldundef{userd}{%
      %
    }{%
      \textbf{\\\color{BrickRed}\printfield{userd}.}
    }
  }
}

% https://tex.stackexchange.com/questions/297087/putting-the-title-first-in-the-bibliography
\newcommand{\nameuse}[1]{%
  \def\do##1{\settoggle{blx@use##1}{#1}}%
  \dolistcsloop{blx@datamodel@names}}

\newcommand{\nameusesave}{%
  \def\do##1{%
    \providetoggle{blx@save@use##1}%
    \iftoggle{blx@use##1}{\toggletrue{blx@save@use##1}}{\togglefalse{blx@save@use##1}}%
  }%
  \dolistcsloop{blx@datamodel@names}}

\newcommand{\nameuserestore}{%
  \def\do##1{%
    \iftoggle{blx@save@use##1}{\toggletrue{blx@use##1}}{\togglefalse{blx@use##1}}%
  }%
  \dolistcsloop{blx@datamodel@names}}

\begin{document}

\nocite{*}

\begin{table}
  \centering
  \begin{tabular}{c}
    \textsc{\LARGE Pat Pannuto} \\
    \\
    \textsc{\large Dec 6, 2022}
  \end{tabular}
\end{table}

\begin{table*}
  \centering
  \begin{tabular*}{\textwidth}{l @{\extracolsep{\fill}} r}
    3202 EBU3                          & \href{tel:+18588222924}{Tel: +1.858.822.2924} \\
    9500 Gilamn Dr                     & \href{mailto:ppannuto@ucsd.edu}{ppannuto@ucsd.edu} \\
    San Diego, CA 92093                & \url{https://patpannuto.com} \\
  \end{tabular*}
\end{table*}



%\section*{Research Interests and Overview}
%
%\begin{itemize}
%  \item[]
%    From mainframes to wearables Bell's law captures the march of progress,
%    noting the emergence of a new computing class roughly every decade.  My
%    research explores what will be required to keep enabling these next
%    generations of computing, how insights from the emerging centimeter-scale
%    and nascent millimeter-scale computing class can solve problems across all
%    domains, and how our expectations and interactions with technology will
%    shift as we begin to realize the ubiquitous computing vision.
%
%  \item[]
%    The \textbf{MBus} project considers this from an architectural
%    perspective, finding fundamental area and energy constraints in current
%    interconnect technologies and demonstrating that shifting system
%    management into the interconnect can simplify both system and circuit
%    design.
%
%    \textbf{Slocalization}, the first FCC-compliant ultra wideband backscatter
%    platform, is motivated by deployment challenges, giving one answer for how
%    we might manage the deployment of millions of miniature sensors.  It
%    demonstrates decimeter-accurate sub-microwatt whole-room concurrent
%    localization, creates a novel integration technique to recover signals
%    from far below the noise floor, and introduces the energy versus latency
%    tradeoff for systems design.
%
%    \textbf{Harmonium} explores how to efficiently design an active ultra
%    wideband tag, empowering opportunistic high-fidelity tracking, and first
%    introduced the bandstitching technique that allows access to the ultra
%    wideband channel using widely-available narrowband frontends.
%
%    \textbf{SurePoint} investigates diversity in the ultra wideband channel,
%    efficient protocols to capture multiple independent samples, and is the
%    first to demonstrate constructive interference with 802.15.4a.
%
%    The \textbf{Tock} operating system answers questions of management,
%    bringing proper process isolation to embedded systems via new hardware and
%    language features that afford safety. Tock addresses fundamental
%    robustness and adaptability tensions with the introduction of grants, a
%    mechanism for a statically allocated kernel to safely perform dynamic
%    allocations in process memory.
%
%    \textbf{Luxapose} shows how the smartphone camera can recover data from
%    visible lights using the rolling shutter effect and realize
%    centimeter-accurate position and single-degree accurate orientation from
%    the projection of these lights on the camera imager.
%
%    \textbf{Opo} crafts a novel, highly efficient ultrasonic wakeup
%    frontend that enables a new broadcast ranging primitive, affording
%    infrastructure-free human interaction tracking with high spatio-temporal
%    fidelity.
%
%    %aiming to understand how our interaction and utilization of technology
%    %will shift as computing becomes omnipresent and its operation and
%    %interaction shifts from conscious action to unconscious extension of
%    %perception and ability.
%    %What advancements will most change how people interact with themselves,
%    %the world, and one another, and what innovations facilitate these paradigm
%    %shifts?
%\end{itemize}


\section*{Academic Appointments}

\begin{itemize}
  \item[]
    \textbf{University of California San Diego}, San Diego, CA (2019--present) \\
    Assistant Professor, Computer Science Engineering
\end{itemize}

\section*{Research Interests}
\begin{itemize}
  \item[] Embedded Systems, Computer Architecture, Wireless Communications, Mobile Computing, Operating Systems, and Development Engineering
\end{itemize}

\section*{Education}

\begin{itemize}
  \item[]
    \textbf{University of California, Berkeley}, Berkeley, CA (2017--2020) \\
    Ph.D. in Electrical Engineering and Computer Sciences \\
    Advisor: Prabal Dutta

  \item[]
    \textbf{University of Michigan}, Ann Arbor, MI (2012--2017) \\
    M.Eng. in Computer Science \\
    Advisor: Prabal Dutta

  \item[]
    \textbf{University of Michigan}, Ann Arbor, MI (2007--2012) \\
    B.S.Eng. in Computer Engineering
\end{itemize}



\input{gen/awards.tex}



\section*{Advising and Mentoring}

\renewcommand{\arraystretch}{0.5}
\begin{tabular}{>{\bf}p{1cm} l}
  2018 & \makecell{
    \href{https://n.ethz.ch/~abiri/}{Andreas Biri}, (M.Sc.; went on to Ph.D. program at ETH Z\"urich)\\
    Thesis: ``TotTernary: A wearable platform for social interaction tracking''
  } \\

  \\

  2014 & \makecell{
    Noah Nuechterlein, (undergraduate independent study): Applied computer vision
  } \\
\end{tabular}
\renewcommand{\arraystretch}{1.0}

\section*{Teaching Experience}

\renewcommand{\arraystretch}{0.5}
\begin{tabular}{>{\bf}p{2.1cm} l}
  \makecell*[{{p{2.1cm}}}]{
    \raggedright
    2020 Fall
  } & \makecell{
    \textbf{Primary Instructor},
    \href{https://patpannuto.com/classes/2020/fall/cse141/}%
    {CSE 141: Introduction to Computer Architecture}
  } \\

  \makecell*[{{p{2.1cm}}}]{
    \raggedright
    2020 Fall
  } & \makecell{
    \textbf{Primary Instructor},
    \href{https://sites.google.com/eng.ucsd.edu/embeddedlunch/}%
    {CSE 290: Seminar on Topics in Embedded Systems}
  } \\

  \makecell*[{{p{2.1cm}}}]{
    \raggedright
    2020 Winter
  } & \makecell{
    \textbf{Primary Instructor},
    \href{https://patpannuto.com/classes/2020/winter/cse291/}%
    {CSE 291: Platforms \& Systems to Bridge the Digital \& Physical World}
  } \\

  \makecell*[{{p{2.1cm}}}]{
    \raggedright
    2016 Fall
    2016 Winter
  } & \makecell*[{{p{5.5in}}}]{
    \textbf{Primary Instructor}, EECS\,398: Computing for Computer Scientists \\[.5em]
        A new class designed and built from scratch. This class attempts
        to address the experience gap that exists across the spectrum of
        incoming Computer Science students. While driven by tools (shells,
        build systems, debuggers, version control), it explores how and why
        computer scientists interface with computers differently in their
        day-to-day activities, how to apply principles learned in courses to
        everyday activities, and ultimately how to be a more efficient user of
        computing resources.  \\[5.5em]
        This course has been adopted as part of the permanent curriculum
        at the University of Michigan as EECS\,201: Computing Pragmatics, an
        advised co-requisite for first-year EECS majors. \\[1.5em]
        \url{https://c4cs.github.io} \\[.5em]
        \emph{In 2017, I was awarded the Rackham Graduate School Outstanding Graduate Student Instructor and the College of Engineering Richard \& Eleanor Towner Prize for Outstanding Graduate Student Instructors for this course.} \\
  } \\

  \\

  \makecell*[{{p{2.1cm}}}]{
    \raggedright
    2015 Fall
    2015 Winter
  } & \makecell{
    \textbf{Graduate Teaching Assistant}, EECS\,373: Design of Microprocessor Based Systems
  } \\

  \\

  \makecell*[{{p{2.1cm}}}]{
    \raggedright
    2012 Winter
  } & \makecell{
    \textbf{Undergraduate Teaching Assistant}, EECS\,470: Computer Architecture (W12)
  } \\

  \\

  \makecell*[{{p{2.1cm}}}]{
    \raggedright
    2012 Winter
    2011 Fall
    2011 Winter
    2010 Fall
  } & \makecell{
    \textbf{Undergraduate Teaching Assistant}, EECS\,482: Introduction to Operating Systems
  } \\

  \\

  \makecell*[{{p{2.1cm}}}]{
    \raggedright
    2011 Fall
    2011 Winter
  } & \makecell{
    \textbf{Undergraduate Teaching Assistant}, EECS\,373: Design of Microprocessor Based Systems
  } \\
\end{tabular}
\renewcommand{\arraystretch}{1.0}


\section*{Professional Service}

\renewcommand{\arraystretch}{0.5}

\begin{longtable}{>{\bf}p{2.1cm} l}
  2021          & \makecell{The 20th International Conference on Information Processing in Sensor Networks (IPSN) -- TPC Member} \\
  \\

  2020          & \makecell{8th International Workshop on Energy Harvesting \& Energy-Neutral Sensing Systems (ENSsys) -- TPC Member} \\
  \\

  2020          & \makecell{The 3rd Workshop on Benchmarking Cyber-Physical Systems and Internet of Things (CPS-IoTBench 2020) -- TPC Member} \\
  \\

  2019          & \makecell{ACM Workshop on Data Acquisition to Analysis (DATA\,19) -- Conference Organizer \& PC Chair} \\
  \\

  2019--present & \makecell{Recurring reviewer for IEEE/ACM Transactions on Networking (TNET)} \\
  \\

  2018          & \makecell{ACM Workshop on Data Acquisition to Analysis (DATA\,18) -- TPC Member} \\
  \\

  2014--2016    & \makecell{Recurring reviewer for IEEE Transactions on Mobile Computing (TMC)} \\
  \\

  2015--2016    & \makecell{Computer Science Engineering Student Faculty Representative} \\
  \\

  2013--2015    & \makecell{Computer Science Engineering Graduate Student Body President} \\
  \\

  2015          & \makecell{Recurring reviewer for USAID Development Innovation Ventures (DIV)} \\
  \\

  2014          & \makecell{ACM Workshop on Visible Light Communication Systems -- Demo Co-Chair} \\
  \\

  2013--2014    & \makecell{Recurring reviewer for IEEE Transactions on Circuits and Systems II (TCAS-II)} \\
  \\
\end{longtable}

\renewcommand{\arraystretch}{1.0}



\section*{Invited Presentations}

\renewcommand{\arraystretch}{0.5}
\begin{longtable}{>{\bf}p{1cm} l}
  2020 & \makecell{
    \textbf{Panel: Benchmarking IoT for social distancing solutions} \\
    The 3rd Workshop on Benchmarking Cyber-Physical Systems and Internet of Things (CPS-IoTBench 2020) \\
    \textbf{\color{BrickRed} Invited Panelist} \\
  } \\

  \\

  2019 & \makecell{
    \textbf{Planes, Trains, Apples, and Oranges: Reproducible Results and Fair Comparisons in Localization Research} \\
    2nd Workshop on Benchmarking Cyber-Physical Systems and Internet of Things (CPS-IoTBench'19) \\
    \textbf{\color{BrickRed} Invited Talk} \\
  } \\

  \\

  2019 & \makecell{
    \textbf{A Modular Platform for Nanopower Computing} \\
    IBM Research; Yorktown Heights, New York \\
    ETH Z\"urich; Z\"urich, Switzerland \\
    University of Michigan; Ann Arbor, Michigan \\
    University of California, Los Angeles; Los Angeles, California \\
    University of Wisconsin-Madison; Madison, Wisconsin \\
    Cornell University; Ithaca, New York \\
    Princeton University; Princeton, New Jersey \\
    Carnegie Mellon University; Pittsburgh, Pennsylvania \\
    Massachusetts Institute of Technology; Cambridge, Massachusetts \\
    University of Washington; Seattle, Washington \\
    University of California, San Diego; San Diego, California \\
    Northwestern University; Evanston, Illinois \\
    \textbf{\color{BrickRed} Invited Talk} \\
  } \\

  \\

  2016 & \makecell {
    \textbf{MBus: A power-aware interconnect for ultra-low power micro-scale system design} \\
    DARPA Near Zero Power RF and Sensor Operations (N-ZERO) Program Review; Santa Barbara, California \\
    \textbf{\color{BrickRed} Invited Talk} \\
  } \\

  \\

  2016 & \makecell{
    \textbf{Ultra Wideband and Indoor Localization} \\
    3rd ACM Workshop on Hot Topics in Wireless (HotWireless'16); New York City, New York \\
    \textbf{\color{BrickRed} Invited Talk}
  } \\

  \\

  2016 & \makecell{
    \textbf{The Recent Past and Distant Future of [Micro-Scale] Embedded Systems} \\
    NextMote: Next Generation Platforms for the Cyber-Physical Internet, part of the International Conference on Embedded Wireless Systems and Networks (EWSN'16); Graz, Austria \\
    \textbf{\color{BrickRed} Keynote Address} \\
  } \\

  \\

  2016 & \makecell{
    \textbf{PolyPoint and the First Steps Towards Ubiquitous Localization} \\
    Student Summit on Mobility, Systems, and Networking, Microsoft Research; Petaluma, California \\
  } \\

  \\

  2015 & \makecell{
    \textbf{Sensor Systems and the Art of Effectively Deploying Sensor Networks} \\
    TechChange TC111: Technology for Monitoring and Evaluation; Online \\
    \textbf{\color{BrickRed} Invited Guest Speaker} \\
  } \\

  \\

  2014 & \makecell{
    \textbf{Embedded System Design and the Internet of Things} \\
    Stanford Internet of Things Industrial Research Program; Stanford, California \\
    \textbf{\color{BrickRed} Invited Talk} \\
  } \\

  \\

  2014 & \makecell{
    \textbf{Sensing Technologies for Data Collection and Monitoring} \\
    State of the Science, Development Impact Lab (DIL) and USAID Higher Education Solutions Network (HESN); Washington, D.C. \\
    \textbf{\color{BrickRed} Invited Talk} \\
  } \\

  \\

  2013 & \makecell{
    \textbf{MBus: Enabling the Next Generation of Sensors and Systems} \\
    TerraSwarm Annual Meeting; Berkeley, California \\
  } \\
\end{longtable}

\renewcommand{\arraystretch}{1.0}



\section*{References}

\begin{tabularx}{\textwidth}{X X X}
  \textbf{\href{https://people.eecs.berkeley.edu/~prabal/}{Prabal Dutta}} &
  \textbf{\href{https://users.ece.cmu.edu/~agr/}{Anthony Rowe}}           &
  \textbf{\href{https://blaauw.engin.umich.edu/}{David Blaauw}}           \\
  %Associate Professor   & Associate Professor        & Professor               \\
  UC Berkeley           & Carnegie Mellon University & University of Michigan  \\
  \href{mailto:prabal@berkeley.edu}{prabal@berkeley.edu}   &
  \href{mailto:agr@ece.cmu.edu}{agr@ece.cmu.edu}           &
  \href{mailto:blaauw@umich.edu}{blaauw@umich.edu}         \\[.5em]
  550C Cory Hall        & CIC 2312                   & 2417C EECS              \\
  UC Berkeley           & 4720 Forbes Ave            & 1301 Beal Ave           \\
  Berkeley, CA 94720    & Pittsburgh, PA 15213       & Ann Arbor, MI 48109     \\
  +1.510.664.9004       & +1.412.268.4340            & +1.734.763.4526         \\
  \\
  \textbf{\href{http://csl.stanford.edu/~pal/}{Philip Levis}} &
  \textbf{\href{https://amitlevy.com/}{Amit Levy}}            \\
  %Associate Professor   & Assistant Professor        \\
  Stanford University   & Princeton University       \\
  \href{mailto:pal@cs.stanford.edu}{pal@cs.stanford.edu}**          &
  \href{mailto:aalevy@cs.princeton.edu}{aalevy@cs.princeton.edu}    \\
  ~\emph{**Please send reference}\\
  ~\emph{solicitations to Ann Harara}\\
  ~\emph{\href{mailto:ann1083@stanford.edu}{ann1083@stanford.edu}}\\[.5em]
  409 Gates Hall        & 307 Computer Science       \\
  Stanford University   & 35 Olden Street            \\
  Stanford, CA 94305    & Princeton, NJ 08540        \\
  +1.650.725.9046       & +1.609.258.8701            \\
\end{tabularx}




% Create "refsections" for each type; label controls active suite of refs
% https://mirrors.concertpass.com/tex-archive/macros/latex/contrib/biblatex/doc/biblatex.pdf
% §3.8.10 Bibliograpy Commands :: Reference Contexts

% multibib style labels with biblatex, from
% http://tex.stackexchange.com/questions/29780/
\nameusesave
\nameuse{false}

\newrefsection[CATbook]
\newrefcontext[labelprefix={B}]
\nocite{*}
\printbibliography[heading=subbibliography,title={\Large Books \& Book Chapters}]

\newrefsection[CATjournal]
\newrefcontext[labelprefix={J}]
\nocite{*}
\printbibliography[heading=subbibliography,title={\Large Journal Publications}]

\newrefsection[CATconference]
\newrefcontext[labelprefix={C}]
\nocite{*}
\printbibliography[heading=subbibliography,title={\Large Conference Publications}]

\newrefsection[CATworkshop]
\newrefcontext[labelprefix={W}]
\nocite{*}
\printbibliography[heading=subbibliography,title={\Large Workshop Publications}]

\newrefsection[CATposterdemo]
\newrefcontext[labelprefix={PD}]
\nocite{*}
\printbibliography[heading=subbibliography,title={\Large Posters and Demos}]

\nameuserestore


%{\bf {\em Talks and Lectures}}
%
%\begin{enumerate}
%
%\item ``Sensing Technologies for Data Collection and Monitoring''. Invited
%  Talk, State of the Science, Georgetown. March 2014
%\item ``An Introduction to Git''. Invited Lecture, Michigan Hackers,
%University of Michigan. April 2012
%
%\end{enumerate}

\end{document}
